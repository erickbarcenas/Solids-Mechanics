
\chapter{RESULTADOS Y CONCLUSIONES}

\section{Análisis y ponderación de resultados \\de con \%E <10\%.}


En este \href{https://bit.ly/37uKJzo}{link está nuestro proyecto interactivo} : se pueden modificar valores y todo se recalcula nuevamente.
\\La versión final de nuestro trabajo estará disponible es este \href{https://erickbarcenas.github.io/Polipasto/}{sitio web}.

\subsection{Pre-conclusiones}
Este proyecto nos ayudó a trabajar los conceptos que estudiamos a lo largo del curso tales como esfuerzo de diseño, cargas, deformaciones etc., de esta forma nos quedó clara la aplicación de cada uno. Por otro lado, nos sirvió también para ver la magnitud que conlleva al trabajar en un problema de diseño desde escoger materiales hasta la manufactura del proyecto, independientemente de que aplicación se le vaya a dar.
Nos deja preparados para afrontar problemas de este tipo en el futuro a lo largo de nuestra carrera, al menos para tener una idea de cómo plantearlo y trabajar en él.
También observamos que al tratarse de un polipasto que debe soportar cierta carga se requiere tener cuidado con los cálculos ya que ligeras diferencias se ven reflejadas en la maqueta, por lo que son de gran importancia ya que la industria puede significar tiempo y dinero.
A lo largo del proceso de realización de este polipasto no hemos topado con demasiados problemas que se han resuelto poco a poco, como son, la elección del material, y sin duda la elección de un motor que se pueda caracterizar. 

Los polipastos son una herramienta que facilita la convención de elementos de gran peso sin la necesidad de utilizar un gran despliegue de fuerza; En nuestro caso está hecho con una transmisión, hecha de piñones de bicicleta. Al realizar el polipasto en físico nos hemos enfrentado a problemas como la soldadura y el armado de la estructura ya que no se cuentan con los conocimientos necesarios para no echar a perder el material, sin embargo, siempre podemos pedir ayuda y resolver los problemas.

Al investigar sobre su funcionamiento nos hemos dado cuenta de que los polipastos han sido una valiosa invención que ha contribuido enormemente en la evolución industrial, sin comentar que las investigaciones actuales están llevando a la creación de polipastos tan avanzados que la carga de un edificio entero dejara de ser un sueño dentro de algunos años, y pasará a ser una realidad gracias a los nuevos materiales que se están implementado en la investigación de los polipastos en este momento.

